\documentclass[12pt]{article}

\usepackage{setspace}
\usepackage[letterpaper,margin=1in]{geometry}
\usepackage{verbatim}

\title{Nabi - A Preventative Solution Against JavaScript and Scriptless Tabnabbing Attacks}
\author{Michael Baraboo, William Schattgen, Christopher To}
\date{4 May 2015}
\begin{document}
\maketitle

\thispagestyle{empty}

\begin{abstract}
  
\end{abstract}

\begin{doublespace}

\section{Introduction}
\begin{comment}
The use of the internet has become an almost unavoidable part of life. As the internet becomes more involved with everyday life, users put more and more of their personal information online. Phishing is the process of stealing, gaining access to, or recording personal accounts or information. Phishing attacks try to convince the target to give up personal information such as passwords, account names, credit card numbers, addresses, and other valuable information or credentials. Phishing has many mediums such as email, phone, websites, and even mail. This information can be used in a variety of destructive ways from the thievery of personal assets to identity theft. Tabnabbing is a phishing technique that relies on the user’s trust of websites they have already viewed and considered safe. The site is changed to a malicious site when the user is not viewing the page. This is done using JavaScript to change the page while the tab is not in focus or by using the meta tag with a timer to redirect the page at a time when the user is most likely not viewing it. Phishing attacks are becoming more common as web use increases, and as users become more aware of phishing attacks, phishing techniques adapt to continue to be effective.
\end{comment}

\section{Background}
\begin{comment}
Phishing attacks are not a new occurrence; they have been around since there has been personal data or credentials to steal. However, ``tabnabbing'' is a relatively recent addition to the list of phishing techniques, coined by Aza Raskin in 2010 \cite{Raskin}. Tabnabbing is a phishing technique that takes advantage of two commonly exhibited user behaviors for modern browser usage. The first is that users often open several tabs in a browser window and navigate between them without closing previous tabs. The second is the user tendency to leave browsers open for extended periods of time while away from their computers in order to access the pages more easily upon their return. 
\end{comment}

\subsection{TabShots}
\begin{comment}
One approach to combating tabnabbing is an extension known as TabShots, a program created by Philippe De Ryck, Nick Nikiforakis, Lieven Desmet, and Wouter Joosen \cite{TabShots}. TabShots continuously takes screenshots of tab and compares them to the previous ones and highlights any changes in red yellow or green depending on how many changes were made. When the user navigates back to a tab that has been altered in a tabnabbing attack, they will see the highlighted portions of the screen and receive a notification that the tab has been altered. 
Performance inpact on larger screens
Relies on user to make decision
\end{comment}

One approach to combating tabnabbing is a Google Chrome browser extension known as TabShots, which was created by Philippe De Ryck, Nick Nikiforakis, Lieven Desmet, and Wouter Joosen \cite{TabShots}.  To detect whether a tab has changed or not while not in focus, TabShots uses visual comparison of the tab's appearance before and after the change in the user's focus.  The extension records screenshots of the currently focused tab at regular intervals, using a simple Google Chrome API call and storing the screenshot as a data URL.  The program also keeps record of the tab's favicon, or the icon displayed in the tab's url and title space.  These two sets of images provide the basis for comparison when a tab regains focus.

When any tab regains focus, TabShots compares a new screenshot of the tab and it's current favicon to the most recent stored screenshot and favicon that was recorded before the tab lost focus.  The small favicons are compared by source, and the screenshots are compared in a visually divided manner using the HTML5 canvas element.  The screenshots are divided in a raster of fixed-size tiles, and each tile in the new screenshot is compared to the most recently stored screenshot of the tab.  The tiles are sized as 10x10 pixels, which are deemed by the authors as a balance between performance and precision.  If the tiles do not match exactly, that area is marked as changed.  The HTML5 canvas element provides powerful image manipulation capabilities that allowed the rastering and comparison algorithms to be implemented.

Once the differences in the current tab from its previous version are calculated, an overlay is injected into the page that shows the differences in semi-transparent red.  The overlay is both transparent in both visibility and mouse events, so no mouse and keyboard events will be affected it.  The overlay is constantly checked for its presence by the extension whenever an element is removed, ensuring that a malicious page will not be able to remove the overlay without notifying the extension and user.  TabShots also places an icon onto the browser's toolbar that changes color in response the amount of changes on the currently focused tab.  The icon serves as an unobtrusive security layer that will remain in view should the overlay be removed.

Finally, there is also an intended component for blacklisting sites that have confirmation of tabnabbing attempts.  The developers have created an optaionl server-side component which can pull reported URLs from individual users, and add the confirmed URLs to a database.  If the user decides that the current page is attempting tabnabbing and provides their explicit approval, the application can be signaled to send the accused page's URL, the image before the user switched tabs, and the image of the page after the user switched tabs.  The server-side application uses its own automated comparison process and the further verification of a human analyst to decide if the page is indeed a phishing page, and whether the URL should be added to a subscription model blacklist.  

According to the testings of the creators, TabShots is said to provide large compatibility with most major sites on the internet at a seemingly acceptable processing speed.  The extensions was used on the top 1000 websites listed on Alexa.com, and the results showed that 78\% of these sites fall within the safe threshold of 5\% or less in changed blocks, meaning no compatability issues exist with that page.  About 19\% of these sites have moderate changes less than 40\%, which is the threshold for high amounts of changes, and 3\% report more than 40\% differences in changed blocks.  While the last two numbers may seem high in terms of false positives, the authors note that TabShots never interferes with the functionality of a page, and does provide a whitelist to providing overlays for these sites.  In terms of speed, testing showed that TabShots was able to prepare and calculate the image comparisions on a $1366 \times 768$ window within an average time of 284ms, which should provide little impact on browser operating time.

There are some problems with this application.  First, the performance testing of TabShots was done on a $1366 \times 768$ window.  Many users today have monitors with much larger resolutions such as $1920 \times 1080$ or $2560 \times 1440$, which provide two or four times as many pixels.  This means that TabShots will run considerably longer on these types of monitors and provided a much larger performance strain.  TabShots is designed to calculate the comparisons every time the user changes tab, which lead to very frequent occurrences of calculations depending on user behavior, and can further exaggerate TabShot's slower performance on higher resolution monitors.  Finally, the visual comparison property of TabShots leads to many occurrences of false positives that lead to the high numbers mentioned above.  Examples of non-malicious behavior that would trigger overlay occurrences would be videos, GIFs, image slideshows, overlays, and other dynamic advertisements.  The authors note that slowly images and files would flag large amounts of tile changes.  In addition, if an element was added or removed from the page, every other element on the page would shift, which would flag a major change in the picture comparison algorithm.  The uncertain performance of TabShots on large monitors or frequently changing tabs, as well as its high opportunities for false positives make TabShots a potentially unideal solution to tabnabbing. 
\subsection{TabSol}
\begin{comment}
Another defense against tabnabbing, TabSol created by Amandeep Singh and Somanath Tripathy \cite{TabSol}, compares the hash values of the web page to detect changes and notifies the user. 
\end{comment}

Another attempt to detect tabnabbing is TabSol, a browser plugin created by Amandeep Singh and Somanath Tripathy \cite{TabSol}.  TabSol uses a whitelist and hash computation to determine if a web page has changed while out of focus.  TabSol attempts to detect tabnabbing using less computations and storage than other programs such as TabShots.

The process of TabShots begins with a simple whitelisting check, to check whether the newly opening URL has already been verified as legitimate.  If so, the process execution is stopped.  Otherwise, TabSol will calculate the hash digest of the web page source to save the current state of the page, using the cryptographic hash function SHA-1.  When the user switches back to the old tab, TabSol will recompute the hash digest of the page, and compare it to the stored digest.  If they match, TabSol determines that the site is legitimate and adds the page into the domain whitelist mentioned above.  Else, the page is treated as suspicious, and TabSol attempts to find a login form on the changed page, as the authors consider its presence necessary in a phishing attack.  The authors chose to sue the login form detction method presented in CANTINA+.  If a login form is present, then the page is considered a phishing web page attempting tabnabbing, and appropriate actions can be taken.

The performance of this implementation is a strong feature of this browser extension.  On an average machine with a 2.53 GHz processor, 4GB RAM and Windows 7 (64-bit), TabSol takes on average 157ms to detect a Tabnabbing attack, if it exists.  The process only uses two hash computations, and the most time-consuming part is the login form detection, which only is used on a subset of tab changes where the hash computations do not match.  In addition, the hash values are the only things needing to be stored, which takes up much less memory than screenshots of the user's windows in the case of TabShots.  Finally, the authors make the case that the login form detection is a crucial component of TabSol that is lacking in other tabnabbing detection systems, and prevents the large amounts of false positives in TabSol tabnabbing detection.

The authors do leave some issues of TabSol unaddressed in their paper.  First, the authors make the fundamental assumption that any tabnabbing page will have a login form.  This can be simply avoided by using a tabnabbing page that uses alternative means of communication such as a phone number, email address, or even a link to another login page.  The login form detection system will completely fail in detecting these types of phishing attacks.  Second, TabSol immediately and prematurely place any site that has matching hashes onto the whitelist, which could result in malicious sites being placed onto the whitelist.  If the user switches away and back to the tab before the malicious tab is scheduled to change, the tab will not change at all, and it will pass TabSol testing and be placed on the blacklist.  This login form basis and premature whitelisting can easily lead to the infiltration of tabnabbing sites.

\subsection{NoTabNab}
\begin{comment}
Another extension, NoTabNab created by Seckin Anil Unlu and Kemal Bicakci \cite{NoTabNab}, notifies the user when the favicon or the tab name change as well as when a new layer is added to an inactive tab. 
\end{comment}

NoTabNab is a browser addon designed by Secking Anil Unlu and Kemal Bicakci.  Their main objective was to focus on the favicon, page title, and layout changes for each tab, as the authors state that these properties must change in a tabnabbing attack.  In particular, the program looks at important offset and margin values that are calculated for the page after the CSS style rules and properties are applied to each HTML element.  The authors argue that any tabnabbing site must have some influence on these values, and thus they can be used to detect if the site has changed in an attempt of a tabnabbing attack.

The application begins by recording the page titles and favicons for each tab.  NoTabNab then records the important layout values of the topmost page elements on these tabs.  By topmost elements, the authors mean the elements which have the highest value on the z-axis above all other elements, as these are the elements that are most likely to change in a tabnabbing attack.  When the tab focus returns to a tab, the stored values of the tab are compared to the new ones, and if any change is found, the user is passively alerted through highlighting the address bar.  It should also be noted that if the page refreshes itself, redirects to somewhere else, or loads a new URL, the entire recording process is repeated and compared to the older recorded state. 

The authors point out the advantages of their minimal information gathering methodology.  The primary benefits of only recording data about the topmost elements is the reduced work over gathering the necessary data points for every element, which results in less storage and processing to initially store the data, and less processing to compare the various data points for each element in the document.  In addition, the comparison of values will be resource friendly, as only the simple favicon, title, and important CSS values are compared for each page, as opposed to entire high resolution pictures or entire documents.  

The less thorough and structured methodology of NoTabNab does allow some problems in usability and tabnabbing tracking.  The elements that are tracked by NoTabNab are chosen at random by finding the topmost elements of the page at random grid points.  NoTabNab may be intentionally tricked into tracking nonchanging elements by placing very small, almost invisible elements that are only a handful of pixels large on the page.  If these elements are scattered through the page at the highest z-order, then NoTabNab will track their changes, and not the changes of any other elements.  In addition, if the document contains many iframes within each other, then the addon must recursively record and check the elements inside them, increasing the storage and processing requirements of the application.  Finally, if the user resizes their window, any tab that resizes or changes their layout as a direct result of that action will appear to have radically changed under NoTabNab, and it will provide entirely false and ignorant data about the changes of the user's tabs.

\section{Proposal and Methodology}
\begin{comment}
Each of these approaches has advantages and disadvantages, but one particular disadvantage all these methods possess is that they are all only detection programs. We believe that a preventative method is a more effective way to combat tabnabbing.
We propose a web browser extension to prevent attackers from executing their tabnabbing attacks. Aza Raskin suggested the use of the NoScript add-on for the Firefox web browser to eliminate the use of JavaScript across all tabs in the browser.  Our proposed system is similar to this approach, but is implemented in the Google Chrome web browser and differs from Raskin’s proposed solution in some key elements. Unlike NoScript, which blocks all javascript across every tab within a browser window, our extension will block javascript specifically in all non-active tabs that are not part of a trusted whitelist.  
\end{comment}
Each of these proposed approaches has particular advantages and disadvantages, but all of them require that the tabnabbing attack be detected first and then dealt with by the software or by alerting the user. Our proposed solution takes a preventative approach to stopping tabnabbing attacks. Aza Raskin suggested the use of the NoScript add-on for the Firefox web browser to eliminate the use of JavaScript across all tabs in the browser. Our proposed system, Nabi, is similar to this approach, but is implemented as an extension in the Google Chrome web browser and differs from Raskin’s proposed solution in some key elements. Unlike NoScript, which blocks all javascript across every tab within a browser window, our extension will block javascript specifically in all non-active tabs.

Our proposal is the prevent all tabnabbing attacks by restricting their ability to execute within the browser. Since tabnabbing attacks inherently run when the user is not looking, our solution is to prevent their execution whenever the user is not viewing that tab. If the attack occurs while the user is looking at the page, they will hopefully notice the change and avoid having their personal data phished by the newly created website. More so, since the active tab is allowed to run javascript, it gives the user the ability to still navigate to and utilize pages that incorporate javascript functionality, unlike Raskin's NoScript suggestion. This can be circumvented via a whitelist feature for both active and inactive tabs, and will be added as a future feature in Nabi.

The primary aspects of Nabi are as follows: Javascript restriction and allowance system, whitelisting of trusted domains, and the detection and prevention of scriptless redirect attacks. 

\section{Results}
The current build of Nabi restricts javascript from running in every tab the user has switched from. The tab is then refreshed to reflect those restrictions and to the cancel the actively running javascript on that page. The page that is switched to is given access to run javascript and is also refreshed to reflect the change in permissions. This prevents javascript in the unactive from running while the user is not viewing it, but allows it to reactivate once the user returns.

The only two forms of tabnabbing are through javascript and via the scriptless HTML meta tag redirect. Our system currently only prevents the former of the two types of attack, but prevents this attack completely. Previous approaches must be able to detect an active attack and take necessary action, and thus can be circumvented if the attacker is aware of the method's metrics of detection. Nabi can provably prevent all javascript tabnabbing attacks before they ever occur by restricting their access to execute only when the user has that tab focused.

However, the scriptless form of the attack can still successfully occur. The current build of Nabi does not detect or block meta tag redirections and 

\section{Conclusion}

\end{doublespace}


\bibliographystyle{plain}
\bibliography{bibliography}

\end{document}
